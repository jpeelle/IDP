
%!TEX TS-program = xelatex

\documentclass[letterpaper,oneside,11pt,article, portrait]{memoir}
\usepackage[pdftitle={Scientific Paper Checklist}, pdfauthor={Jonathan Peelle}, colorlinks=true, urlcolor=blue]{hyperref}


\setlrmarginsandblock{.8in}{.8in}{*}
\setulmarginsandblock{.8in}{.8in}{*}

\usepackage{amsfonts}
\usepackage{array,ragged2e}
\usepackage{fontspec,xunicode}
\defaultfontfeatures{Mapping=tex-text}

\setsecnumdepth{section}

% For tabularx columns (memman.pdf p. 233-234
%\renewcommand{\tabularxcolumn}[1]{\centering{p{#1}}}


\setsecnumdepth{chapter}  % section
\setsecheadstyle{\LARGE}
\setbeforesecskip{.25in}
\setaftersecskip{.5in}
\setsecindent{0in}


\setaftersubsecskip{.3in}

\setromanfont{Helvetica}
\checkandfixthelayout	% for memoir class

\begin{document}
\pagestyle{empty}


\begin{centering}

{\Large Setting Yourself up for Success with an Individual Development Plan}

\small{\url{http://github.com/jpeelle/IDP}}

\end{centering}

\vspace{.2in}


\textbf {Yearly goals} are the core of the IDP. Many training and mentoring programs have IDPs that are reviewed yearly. It's a long enough time period to think about bigger goals, but short enough that you can link it to concrete actions.

\textbf{Long-term goals} should encourage you to think big picture---5 or 10 years is good. (If you aren't used to thinking about long-term goals, this is a good exercise!)

\textbf{Short-term goals} help you translate your yearly goals into time-dependent action. For people on an academic schedule, I find setting semester-length goals (Fall, Spring, Summer) aligns well with the flow of the year. For others, quarterly makes sense.

The idea is that eventually you should see a path from your weekly actions through your long-term goals. If some activity you are asked to do doesn't fit into your big picture, that might be a sign you need to say no.

Review your goals on your short-term schedule (e.g., every semester or quarter). It is normal to not get accomplished all that you wanted to. However, you can use this opportunity to think about ways to improve, and re-evaluate how to best reach your longer-term goals.

Armed with a fully operational IDP you might then consider making \textit{weekly} priorities so that your important big-picture goals don't get lost in the shuffle of life.

As you make your goals:

\begin{enumerate}
\item Consider your whole life. You may not want to share everything with your boss or mentor, but have some version of your IDP that includes your non-work goals (health, relationships, hobbies). If your written goals and plans don't reflect you as a whole person you will get frustrated (and it's harder to make goal-oriented decisions).

\item Work on honing your goal-setting language. Framing goals effectively takes practice and can make a big difference in your success.

\end{enumerate}

In 1981, George Doran wrote ``There's a S.M.A.R.T.\ Way to Write Management's Goals and Objectives'', which inspired many. Michael Hyatt has a modified version he calls \href{https://fullfocus.co/goal-setting/}{SMARTER}, in which goals should be:
\begin{enumerate}

\item Specific. The more specific you are, the easier it is to link actions to your goals.

\item Measurable.

\item Actionable. Every goal should start with an action verb (``quit'', ``finish'', ``write'', etc.). 

\item Risky. A good goal should stretch you (otherwise, you probably wouldn't need to write it down).

\item Time-keyed. ``A goal without a date is just a dream.''

\item Exciting. If you aren't personally excited about a goal it will be difficult to accomplish.

\item Relevant. Goals should align with your other goals and season in life.

\end{enumerate}

Often it works to first come up with a ``regular language'' goal (``Submit manuscript'') and then use that to create SMARTER goals that are what you implement (``write 1 hour per day every weekday'').

Finally, you need to find your own style, but I find using pen and paper to write goals down to be more effective than typing. It forces me to slow down and feels more intentional.


\clearpage

\section{Individual Development Plan}


\noindent \begin{tabularx}{\textwidth}{ l X l X |}
Name:& &Date:& \\
\hline
\end{tabularx}

\subsection{Long-term goals (\hspace{.5in} years)}

\begin{center}
 \begin{tabular}{|p{2.5in}|p{4in}|}
Regular goal (``submit a manuscript'')& SMARTER goal(s) (``spend 1 hour per day writing'') \\
\hline
\vspace{6in} \ & \\
\hline
\end{tabular}
\end{center}



\subsection{Yearly goals (For the year:\hspace{1in})}

\begin{center}
 \begin{tabular}{|p{2.5in}|p{4in}|}
Regular goal (``submit a manuscript'')& SMARTER goal(s) (``spend 1 hour per day writing'') \\
\hline
\vspace{8in} \ & \\
\hline
\end{tabular}
\end{center}

\clearpage


\subsection{Short-term goals (For the period:\hspace{3in})}

\begin{center}
 \begin{tabular}{|p{2.5in}|p{4in}|}
Regular goal (``submit a manuscript'')& SMARTER goal(s) (``spend 1 hour per day writing'') \\
\hline
\vspace{8in} \ & \\
\hline
\end{tabular}
\end{center}

\clearpage



\subsection{Short-term goals (For the period:\hspace{3in})}

\begin{center}
 \begin{tabular}{|p{2.5in}|p{4in}|}
Regular goal (``submit a manuscript'')& SMARTER goal(s) (``spend 1 hour per day writing'') \\
\hline
\vspace{8in} \ & \\
\hline
\end{tabular}
\end{center}

\clearpage



\subsection{Short-term goals (For the period:\hspace{3in})}

\begin{center}
 \begin{tabular}{|p{2.5in}|p{4in}|}
Regular goal (``submit a manuscript'')& SMARTER goal(s) (``spend 1 hour per day writing'') \\
\hline
\vspace{8in} \ & \\
\hline
\end{tabular}
\end{center}

\clearpage




\subsection{Short-term goals (For the period:\hspace{3in})}

\begin{center}
 \begin{tabular}{|p{2.5in}|p{4in}|}
Regular goal (``submit a manuscript'')& SMARTER goal(s) (``spend 1 hour per day writing'') \\
\hline
\vspace{8in} \ & \\
\hline
\end{tabular}
\end{center}

\clearpage

\end{document}